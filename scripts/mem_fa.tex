\documentclass[12pt,a4paper]{article}
\usepackage{xepersian}
\settextfont{B Nazanin}
\setlatintextfont{Times New Roman}
\linespread{1.3}
\hypersetup{hidelinks}

\title{طراحی و پیاده‌سازی نورون‌های نشت‌دار تجمیع‌کننده و شلیک‌کننده (LIF) روی FPGA}
\author{علی یزدان‌پناه\\
دانشکده مهندسی برق و کامپیوتر، دانشگاه شیراز}
\date{}

\begin{document}
\maketitle

\begin{abstract}
در این گزارش، طراحی، شبیه‌سازی و پیاده‌سازی سخت‌افزاری نورون دیجیتال LIF و یک شبکه عصبی اسپایکینگ دو نورونی معرفی می‌شود. پیاده‌سازی روی FPGA سری Artix-7 و با استفاده از Vivado 2017.1 انجام شده و همه محاسبات با قالب ثابت Q4.12 پیاده‌سازی شده‌اند. نورون دیجیتال رفتارهای شاخصی مانند ادغام ورودی، نشت زمانی، آستانه‌گذاری، بازنشانی و دورهٔ مقاومت را بازتولید می‌کند و امکان اعمال تأخیر سیناپسی قابل تنظیم را فراهم می‌آورد. سه محیط آزمون مجزا برای راستی‌آزمایی رفتاری طراحی شده‌است و نتایج نشان می‌دهد که پیاده‌سازی سخت‌افزاری با مدل تحلیلی تطابق دارد. همچنین راهکارهای سنتز و تحلیل منابع برای تراشه XC7A35T ارائه شده تا کاربر بتواند به سادگی طراحی را روی بردهای آموزشی تکرار کند.
\end{abstract}

\section{مقدمه}
مهندسی نورومورفیک در پی آن است که اصول پردازشی سیستم عصبی زیستی را در سخت‌افزارهای الکترونیکی بازآفرینی کند. در این رویکرد، اطلاعات نه به صورت مقادیر پیوسته، بلکه در قالب وقوع رویدادهای گسسته (اسپایک) منتقل می‌شوند و همین امر زمینهٔ ساخت سامانه‌های کم‌مصرف، کم‌تأخیر و مقاوم به ورودی‌های تنک را فراهم می‌آورد \cite{GerstnerKistler2002,Maass1997}. مدل نورون LIF به دلیل سادگی و امکان نگاشت مستقیم در قالب جمع و ضرب‌های ثابت، انتخابی رایج برای پیاده‌سازی دیجیتال روی FPGA است. طی دههٔ گذشته تراشه‌های بزرگ‌مقیاس مانند TrueNorth \cite{Merolla2014}، Loihi \cite{Davies2018} و سامانهٔ SpiNNaker \cite{Furber2014} نشان داده‌اند که محاسبات مبتنی بر اسپایک از نظر مصرف انرژی و مقیاس‌پذیری توجیه‌پذیر است. در محیط‌های آموزشی، FPGAها به خاطر پیکربندی بیت‌به‌بیت، زمان‌بندی قطعی و ابزارهای شبیه‌سازی غنی، بستری مناسب برای آموزش مفاهیم نورومورفیک به شمار می‌آیند.

این پروژه با هدف ایجاد پلی میان نظریه و پیاده‌سازی، نورون LIF و یک شبکهٔ دو نورونی با سیناپس‌های دارای تأخیر را روی FPGA پیاده کرده و مجموعهٔ آزمون‌هایی برای سنجش رفتار طراحی به صورت گام‌به‌گام فراهم آورده‌است. تمام بخش‌ها با کد SystemVerilog قابل حمل نوشته شده و از جریان ابزار Vivado 2017.1 برای سنتز و شبیه‌سازی استفاده شده‌است.

\section{مرور ادبیات}
مدل نورون LIF ریشه در کارهای کلاسیک لاپیک و سپس مدل‌های دقیق‌تر هاجکین–هاکسلی دارد، اما به خاطر فرم خطی–بخش‌بندی سادهٔ خود، برای تحلیل و پیاده‌سازی دیجیتال مناسب‌تر است \cite{GerstnerKistler2002}. مطالعات معاصر نشان می‌دهد که شبکه‌های اسپایکینگ از نظر قدرت محاسباتی با شبکه‌های سنتی قابل رقابت هستند و حتی می‌توان شبکه‌های آموزش‌دیدهٔ ANN را با حفظ دقت به فضای اسپایک تبدیل کرد \cite{Rueckauer2017}. از طرف دیگر، معماری‌های عملی نورومورفیک به وضوح اهمیت سازمان حافظه و مسیریابی رویدادها را نشان داده‌اند \cite{IndiveriLiu2015,LinaresBarranco2011}. در این پروژه تلاش شده است که طراحی ارائه‌شده در عین سادگی، مفاهیم بنیادین این سامانه‌ها مانند نشت، تأخیر سیناپسی، ادغام جریان‌های تحریکی و مهاری و کنترل دورهٔ مقاومت را در خود داشته باشد.

\section{مدل ریاضی نورون LIF}
معادلهٔ پیوستهٔ نورون LIF به صورت
\[
\tau_m \frac{dV}{dt} = -(V - V_{\mathrm{rest}}) + R\,I(t)
\]
بیان می‌شود که در آن $\tau_m$ ثابت زمانی غشا، $R$ مقاومت معادل و $I(t)$ جریان ورودی است. در این پروژه، ورودی‌ها در شکل جریان‌های گسسته با دورهٔ نمونه‌برداری $\Delta t = 10\ \mathrm{ns}$ تزریق می‌شوند و پاسخ نورون به صورت دقیق–گام با ضریب نشت $\alpha = e^{-\Delta t/\tau_m}$ محاسبه شده‌است. زمانی که $V$ از آستانهٔ $V_{\mathrm{th}}$ عبور کند، پالس اسپایک تولید شده و ولتاژ به مقدار $V_{\mathrm{reset}}$ بازگردانده می‌شود و شمارندهٔ مقاومت به مدت مشخصی نورون را در حالت غیرفعال نگه می‌دارد. تمام محاسبات با قالب ثابت Q4.12 انجام و به کمک اشباع، از سرریز جلوگیری می‌شود.

\section{معماری سخت‌افزاری}
سه بلوک اصلی طراحی عبارت‌اند از: (۱) مولد اسپایک پواسونی مبتنی بر LFSR شانزده‌خطی که احتمال وقوع اسپایک در هر چرخه را کنترل می‌کند؛ (۲) هستهٔ نورون LIF با ضرب ثابت–ثابت، مقایسه با آستانه و شمارندهٔ مقاومت؛ و (۳) لایهٔ سیناپسی که تأخیرهای مجزا برای هر ورودی تعریف می‌کند و جریان‌های مهاری/تحریکی را به صورت هندسی مستهلک می‌سازد. همهٔ بلوک‌ها از کلاک مشترک 100~MHz استفاده می‌کنند و ریست فعال-پایین دارند. برای هر نورون، جریان ورودی حاصل از تفاضل دو مسیر $g_E$ و $g_I$ است که وزن‌ها و تأخیرها روی رجیسترها قابل تنظیم‌اند.

\section{آزمون‌های شبیه‌سازی و نتایج}
سه محیط آزمون مجزا تهیه شده‌است:
\begin{description}
  \item[آزمون تک‌نورونه:] نورون با دو ورودی پواسونی مستقل تحریک می‌شود (نرخ ۶.۲۵\% و ۳.۱\%) و خروجی با مدل نرم‌افزاری مرجع مقایسه می‌شود. خطوط غشایی و شمارندهٔ مقاومت با دقت گام به گام منطبق هستند.
  \item[آزمون دو نورونه تصادفی:] شبکهٔ دو نورونه با منابع پواسونی مستقل اجرا می‌شود. نتایج در قالب نمودارهای ولتاژ غشا، اسپایک‌های ورودی و خروجی و نرخ‌های شلیک بررسی شده‌اند و نشان می‌دهند که هم‌زمانی اسپایک‌ها نقش تعیین‌کننده‌ای در شلیک دارد.
  \item[آزمون صحنه‌ای:] چهار سناریوی متوالی شامل گام ثابت، وقفه، سری اسپایک‌های خوشه‌ای و ورودی تصادفی محدود اجرا می‌شود تا رفتار گذرا و بازگشت به حالت سکون مشاهده شود. رسم خطوط $g_E$، $g_I$ و جریان خالص نشانگر پایداری و فیلتر زمانی لایهٔ سیناپسی است.
\end{description}
اسکریپت‌های پایتون موجود در پوشهٔ \lr{scripts} داده‌های CSV تولید شده را رسم می‌کنند و برای هر نمودار، قالب Q4.12 و پارامتر $\alpha=0.96$ در عنوان ذکر شده است.

\section{سنتز و تحلیل منابع}
سنتز OOC در Vivado 2017.1 برای تراشهٔ XC7A35T-CPG236-1 انجام شده است. نتایج اصلی عبارت‌اند از:
\begin{itemize}
  \item مولد پواسونی: ۵ LUT و ۱۷ فلیپ‌فلاپ، بدون DSP یا BRAM.
  \item نورون LIF: یک بلوک DSP48E1 برای ضرب، حدود ۱۰۰ LUT و ۱۰۰ فلیپ‌فلاپ.
  \item لایهٔ سیناپسی دو نورونه: دو DSP48E1، کمتر از ۶۰۰ LUT و بدون استفاده از BRAM.
\end{itemize}
با ترکیب بلوک‌ها برای طرح \lr{snn\_simple}، بهره‌برداری نهایی نیز شامل دو DSP، حدود ۶۰۰ LUT و ۷۰۰ فلیپ‌فلاپ است؛ لذا بیش از ۹۰\% منابع دستگاه همچنان آزاد باقی می‌ماند. تحلیل زمان‌بندی نشان داده است که حتی بدون لوله‌گذاری اضافی، طرح از فرکانس 150~MHz عبور می‌کند و درج یک ثبات بین ضرب و جمع، حاشیهٔ فرکانسی را تا 200~MHz افزایش می‌دهد. از آن‌جا که در مدل بیولوژیکی هر چرخهٔ 10~ns معادل تقریباً ۱~µs در نظر گرفته می‌شود، این حاشیه زمان کافی برای افزودن مسیریاب رویداد، رجیسترهای پیکربندی یا امکانات ثبت داده را فراهم می‌سازد.

\section{بحث و محدودیت‌ها}
پیاده‌سازی ارائه شده با هدف آموزشی ساده‌سازی شده است. قالب Q4.12 دامنهٔ $\pm8$ را پوشش می‌دهد و در صورت افزایش وزن‌های تحریک باید مجدداً پارامترسازی و آزمون شود. وقتی $\alpha$ به ۱ نزدیک شود، دنبالهٔ $g_E$ و $g_I$ دیرتر مستهلک می‌شود؛ در طراحی حاضر برای جلوگیری از رانش، مقادیر نزدیک صفر پس از هر گام صفر می‌شوند. مقیاس‌پذیری به شبکه‌های بزرگ‌تر مستلزم افزودن مسیریاب‌های AER و ساختارهای حافظهٔ سلسله‌مراتبی است. همچنین، مدل LIF فاقد ویژگی‌هایی مانند سازگاری یا اسپایک‌های انفجاری است و برای چنین اهدافی باید مدل‌های پیچیده‌تر مانند AdEx جایگزین شود.

\section{پیشنهاد برای کارهای آتی}
\begin{itemize}
  \item پیاده‌سازی قوانین یادگیری مبتنی بر STDP با استفاده از ضرب‌های اشتراکی و صف رویداد.
  \item افزودن مسیریاب رویداد سبک و واسط پیکربندی برای تنظیم وزن‌ها در حین اجرا.
  \item بررسی قالب‌های عددی گوناگون (Q3.12، Q6.10) و مقایسهٔ تأثیر آنها بر دقت و منابع.
  \item اتصال سامانه به حسگرهای رویدادمحور مانند DVS و اندازه‌گیری انرژی به ازای هر اسپایک.
\end{itemize}

\section{جمع‌بندی}
در این پروژه، نورون LIF و شبکهٔ دو نورونه با تأخیر سیناپسی به صورت سخت‌افزاری و با تکیه بر محاسبات قالب ثابت روی FPGA پیاده شده‌اند. آزمون‌های رفتاری نشان می‌دهد که پیاده‌سازی دیجیتال در سطح چرخه با مدل تحلیلی مطابقت دارد و سامانه می‌تواند ویژگی‌هایی مانند تشخیص هم‌زمانی اسپایک و ویژگی‌های فیلتر زمانی سیناپس را بازتولید کند. نتایج سنتز روی Artix-7 نشان می‌دهد که این طراحی برای محیط‌های آموزشی و پژوهش‌های مقدماتی کاملاً عملی و کم‌هزینه است و می‌تواند پایه‌ای برای توسعهٔ شبکه‌های بزرگ‌تر، قوانین یادگیری درون‌تراشه‌ای و سامانه‌های حسگر–پردازشگر رویداد محور باشد.

\begin{thebibliography}{10}
\bibitem{GerstnerKistler2002}
W.~Gerstner and W.~M. Kistler, \emph{Spiking Neuron Models}. Cambridge University Press, 2002.
\bibitem{Maass1997}
W.~Maass, ``Networks of spiking neurons: The third generation of neural network models,'' \emph{Neural Networks}, vol.~10, pp.~1659--1671, 1997.
\bibitem{Izhikevich2003}
E.~M. Izhikevich, ``Simple model of spiking neurons,'' \emph{IEEE Transactions on Neural Networks}, 2003.
\bibitem{Mead1990}
C.~Mead, ``Neuromorphic electronic systems,'' \emph{Proceedings of the IEEE}, 1990.
\bibitem{Furber2014}
S.~B. Furber \emph{et~al.}, ``SpiNNaker: A million core computing system for real-time brain simulation,'' \emph{Proceedings of the IEEE}, 2014.
\bibitem{Merolla2014}
P.~A. Merolla \emph{et~al.}, ``A million spiking-neuron IC with a scalable communication network,'' \emph{Science}, 2014.
\bibitem{Davies2018}
M.~Davies \emph{et~al.}, ``Loihi: A neuromorphic manycore processor,'' \emph{IEEE Micro}, 2018.
\bibitem{IndiveriLiu2015}
G.~Indiveri and S.-C. Liu, ``Memory and information processing in neuromorphic systems,'' \emph{Proceedings of the IEEE}, 2015.
\bibitem{Rueckauer2017}
B.~Rueckauer \emph{et~al.}, ``Conversion of continuous-valued deep networks to SNNs,'' \emph{Frontiers in Neuroscience}, 2017.
\bibitem{LinaresBarranco2011}
T.~Serrano-Gotarredona and A.~Linares-Barranco, ``AER building blocks for multi-layer multi-chip neuromorphic vision systems,'' in \emph{Proceedings of the IEEE International Symposium on Circuits and Systems (ISCAS)}, 2011, pp.~2797--2800.
\end{thebibliography}

\end{document}
